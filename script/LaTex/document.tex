\documentclass[a4paper,11pt]{book}
% define the title
\author{H. Partl}
\title{Minimalism}
\usepackage{textcomp}
\usepackage{amsfonts}
\usepackage[official]{eurosym}
\begin{document}
	% generates the title
	\maketitle
	% insert the table of contents
	\tableofcontents
	\section{Some Interesting Words}
	Well, and here begins my lovely article.
	\section{Good Bye World}
	\ldots{} and here it ends.
	
	\ldots when Einstein introduced his formula
	\begin{equation}
		e = m \cdot c^2 \; ,
	\end{equation}
	which is at the same time the most widely known
	and the least well understood physical formula.
	\\
	\\
	\ldots from which follows Kirchhoff’s current law:
	\begin{equation}
		\sum_{k=1}^{n} I_k = 0 \; .
	\end{equation}
	Kirchhoff’s voltage law can be derived \ldots
	\\
	\\
	% Example 3
	\ldots which has several advantages.
	\begin{equation}
	I_D = I_F - I_R
	\end{equation}
	is the core of a very different transistor model. \ldots
	
	\today 
	
	\TeX
	
	\LaTeX
	
	\LaTeXe
	
	``Please press the `x' key.''
	
	daughter-in-law, X-rated\\
	
	pages 13--67\\
	
	yes---or no? \\
	
	$0$, $1$ and $-1$
	
	http://www.rich.edu/\~{}bush \\
	http://www.clever.edu/$\sim$demo
	
	It’s $-30\,^{\circ}\mathrm{C}$.
	I will soon start to
	super-conduct.
	
	\texteuro
	
	LM+textcomp \texteuro
	
	\euro
	
	Not like this ... but like
	this:\\ New York, Tokyo,
	Budapest, \ldots
	
	Not shelfful\\
	but shelf\mbox{}ful
	
	H\^otel, na\"\i ve, \'el\`eve,\\
	sm\o rrebr\o d, !‘Se\ norita!,\\
	Sch\"onbrunner Schlo\ss{}
	Stra\ss e
	
	Mr. Smith was happy to see her\\
	cf. Fig. 5\\\frenchspacing
	I like BASIC\@. \frenchspacing What about you?
	
	\section{...}
	\subsection{...}
	\subsubsection{...}
	\paragraph{...}
	\subparagraph{...}
	
	\part{...}
	
	\chapter{introduction}
	
	\emph{If you use
		emphasizing inside a piece
		of emphasized text, then
		\LaTeX{} uses the
		\emph{normal} font for
		emphasizing.}
	
	
	\textit{You can also
		\emph{emphasize} text if
		it is set in italics,}
	\textsf{in a
		\emph{sans-serif} font,}
	\texttt{or in
		\emph{typewriter} style.}
	
	

	\begin{enumerate}
		\item You can mix the list
		environments to your taste:
		\begin{itemize}
			\item But it might start to
			look silly.
			\item[-] With a dash.
		\end{itemize}
		\item Therefore remember:
		\begin{description}
			\item[Stupid] things will not
			become smart because they are
			in a list.
			\item[Smart] things, though,
			can be presented beautifully
			in a list.
		\end{description}
	\end{enumerate}
	
	
	\begin{flushright}
		This text is\\ left-aligned.
		\LaTeX{} is not trying to make
		each line the same length.
	\end{flushright}	

	\begin{flushleft}
	This text is\\ left-aligned.
	\LaTeX{} is not trying to make
	each line the same length.
	\end{flushleft}
	
	
	\begin{center}
		At the centre\\of the earth
	\end{center}
	
	
	\begin{tabular}{|r|l|}
		\hline
		7C0 & hexadecimal \\
		3700 & octal \\ \cline{2-2}
		11111000000 & binary \\
		\hline \hline
		1984 & decimal \\
		\hline
	\end{tabular}
	
	
	\begin{tabular}{|p{4.7cm}|}
		\hline
		Welcome to Boxy’s paragraph.
		We sincerely hope you’ll
		all enjoy the show.\\
		\hline
	\end{tabular}
	
	
	\begin{tabular}{@{} l @{}}
		\hline
		no leading space\\
		\hline
	\end{tabular}
	
	
	\begin{tabular}{l}
		\hline
		leading space left and right\\
		\hline
	\end{tabular}
	
	
	\begin{tabular}{c r @{.} l}
		Pi expression
		&
		\multicolumn{2}{c}{Value} \\
		\hline
		$\pi$
		& 3&1416 \\
		$\pi^{\pi}$
		& 36&46
		\\
		$(\pi^{\pi})^{\pi}$ & 80662&7 \\
	\end{tabular}
	
	
	\begin{tabular}{|c|c|}
		\hline
		\multicolumn{2}{|c|}{Ene} \\
		\hline
		Mene & Muh! \\
		\hline
	\end{tabular}
	
	Add $a$ squared and $b$ squared
	to get $c$ squared. Or, using
	a more mathematical approach:
	$c^{2}=a^{2}+b^{2}$

	\TeX{} is pronounced as
	\(\tau\epsilon\chi\).\\[6pt]
	$100 m^{3}$ of water\\[6pt]
	This comes from my
	\begin{math}\heartsuit\end{math}
	
	
	Add $a$ squared and $b$ squared
	to get $c$ squared. Or, using
	a more mathematical approach:
	\begin{displaymath}
	c^{2}=a^{2}+b^{2}
	\end{displaymath}
	or you can type less with:
	\[a+b=c\]
	
	
	\begin{equation} \label{eq:eps}
	\epsilon > 0
	\end{equation}
	

	
	$\lim_{n \to \infty}
	\sum_{k=1}^n \frac{1}{k^2}
	= \frac{\pi^2}{6}$
	
	
	\begin{displaymath}
	\lim_{n \to \infty}
	\sum_{k=1}^n \frac{1}{k^2}
	= \frac{\pi^2}{6}
	\end{displaymath}
	
	
	\begin{equation}
	\forall x \in \mathbf{R}:
	\qquad x^{2} \geq 0
	\end{equation}
	
	\begin{equation}
	x^{2} \geq 0\qquad
	\textrm{for all }x\in\mathbf{R}
	\end{equation}
	
	\begin{displaymath}
	x^{2} \geq 0\qquad
	\textrm{for all }x\in\mathbb{R}
	\end{displaymath}
	
	$\lambda,\xi,\pi,\mu,\Phi,\Omega$
	
	
	\begin{equation}
	a^x+y \neq a^{x+y}
	\end{equation}
	
	$\alpha \beta \gamma$...
	
	$a_{1}$ \qquad $x^{2}$ \qquad
	$e^{-\alpha t}$ \qquad
	$a^{3}_{ij}$\\
	$e^{x^2} \neq {e^x}^2$
	
	$\sqrt{x}$ \qquad
	$\sqrt{ x^{2}+\sqrt{y} }$
	\qquad $\sqrt[3]{2}$\\[3pt]
	$\surd[x^2 + y^2]$
	
	$\overline{m+n}$
	
	$\underbrace{a+b+\cdots+z}_{26}$
	
	
	\begin{displaymath}
	y=x^{2}\qquad y'=2x\qquad y''=2
	\end{displaymath}
	
	
	\begin{displaymath}
	\vec a\quad\overrightarrow{AB}
	\end{displaymath}
	
	
	
\end{document}